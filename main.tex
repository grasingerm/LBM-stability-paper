\documentclass{article}

\usepackage[affil-it]{authblk}
\usepackage{lineno}
\usepackage[square,numbers,sort&compress]{natbib}
\usepackage{soul}
\usepackage{color}
\usepackage[utf8]{inputenc}

\title{Accuracy, Stability, and Efficiency of Lattice Boltzmann Models in Simulating Non-Newtonian Flow}
%\title{A Comparison of Lattice Boltzmann Method Collision Operators for Accuracy, Stability and Efficiency in Simulating Non-Newtonian Flows}

\author{{Matthew Grasinger}
\thanks{Email: \textit{grasingerm@pitt.edu}}}

\author{John Brigham
\thanks{Email: \textit{brigham@pitt.edu}}}
\affil{Civil and Environmental Engineering Department, University of Pittsburgh}

\begin{document}

\pagenumbering{gobble}
\maketitle
\newpage
\tableofcontents
\newpage
\pagenumbering{arabic}

\linenumbers

Note: anything in {\color{red}red} is just random garbage notes.
You can ignore it for the most part.

\section*{Abstract}

The Lattice Boltzmann method (LBM) is a computational method that can be used for simulating fluid flow.
It is well-suited for complex flows such as non-Newtonian, free surface, and multiphase multicomponent flows.
Non-Newtonian flows are the primary focus of this paper, as many practical engineering problems such as the flow of cement slurries and concretes, the filling of molds by molten plastics, ETC, are best modeled as non-Newtonian fluids, but can lead to numerical instabilities in LBM models.
LBM is a statistical mechanical approach and each time step consists of streaming particle distributions to neighboring nodes, and collisions of particle distributions at each node through a collision operator.
The collision operator is of interest because it has a lot of control over the physics that are simulated, e.g. constitutive laws, interfacial dynamics, etc., and it has implications on numerical stability and computational efficiency.
In this paper, variations of popular collision operators, namely the linear Bhatnagar--Gross--Krook collision operator and the multiple relaxation time collision operator, will be examined for their suitability for simulating non-Newtonian flows in terms of accuracy, numerical stability and computational efficiency.
The structure of the paper will be as follows: blah, blah, blah.

\section{Introduction} %MG: this intro will potentially be long. Should it be broken up into subsections (as it is now)?

\subsection{Applications of Non-Newtonian Flow}

The viscosity of a fluid is a measure of its resistance to deformation.
There are two types of viscosity: (1) dynamic viscosity, or shear viscosity, which is the fluid's resistance to shear stress, and (2) bulk viscosity, or the fluid's resistance to compressive and tensile stresses.
Shear resistance, or dynamic viscosity, is predominately due to friction between fluid particles where as bulk viscosity is due to the packing of fluid particles, the contact that ensues, and more weakly, inter-particle potentials. % MJG: be more precise here.
Outside of applications in sound, vibrations, and shock waves, the dynamic viscosity of a fluid is of primary interest because the flows that are typically studied are shear stress induced.
For many fluids, at a constant temperature, the dynamic viscosity can be considered as constant.
These fluids are known as Newtonian fluids.
Non-Newtonian fluids are fluids in which their apparent dynamic viscosity, or apparent resistance to shear stress, is variable--even at a constant temperature--and can be expressed as a function of strain-rate.
This paper will focus on the flow of non-Newtonian fluids because substances that are best modeled as non-Newtonian fluids are present in many scientific problems and engineering applications, and approximating the flow of non-Newtonian fluids has certain challenges.

There are a number of fluids present in scientific problems and engineering applications that can be classified as non-Newtonian.
To name a few, pastes, slurries, molten plastics, polymer solutions, dyes, varnishes, suspensions, and some biomedical liquids like blood all behave in a non-Newtonian manner \cite{bohme1987non}.
Of all of the different non-Newtonian behaviors that exist, there are two models under which much of the behaviors may be idealized:
\begin{enumerate}
	\item \emph{Yield stress fluids}. Yield stress fluids, also known as Bingham plastics, do not flow until a threshold value of stress, referred to as its yield stress, is exceeded.
	Some examples of yield stress fluids are drilling muds, fresh cement slurries and concretes, and melted metal alloys \cite{ginzburg2002free}.
	Understanding the flow of cement slurries and concretes is important because they are used in the cementing of oil and gas wellbores, and in construction.
	If the cement job for an oil and gas wellbore is not completed successfully then gases and fluids can leak through the wellbore to contaminate ground water or the atmosphere~\cite{grasinger2015simulation}.
	The goal of understanding cement slurry and concrete flows during construction is to understand how construction forms can be completely filled as voids can lead to structural deficiencies~\cite{RARR}.
	Yield stress flow is relevant in many other disciplines and applications because of the many substances that can be modeled as Bingham plastics, e.g. pastes, paints, muds, molten plastics and metals, and blood \cite{wang2011lattice}. %MG: citation is for blood as non-Newtonian
	Non-Newtonian flow is therefor relevant in the engineering and development of pastes and paints, flow and deformation of mud and clay in geotechnical engineering, understanding blood circulation in biomedical applications, materials science and manufacturing of metal and plastic parts, and in particular, the molding and casting of molten metals and plastics.
	
	\item \emph{Power-law fluids}. Power-law behavior is more commonly known as \emph{shear-thinning}--when the apparent viscosity decreases with increasing strain-rate--or \emph{shear-thickening}--when the apparent viscosity increases with increasing strain-rate. Shear-thinning fluids are also known as \emph{pseudoplastics}, some examples include polymer mixes and molten plastics.
	Shear-thickening fluids are also known as \emph{dilatants}, some examples include quicksand, a cornstarch and water mixture, and a silica and polyethylene glycol mixture.
\end{enumerate}

Developing numerical methods for approximating these non-Newtonian flows is important because of how difficult it is, and often impossible it is, to derive analytical, closed-form solutions.

\subsection{Lattice Boltzmann Method for non-Newtonian Flow}

Analytical solutions rarely exists for even the simplest non-Newtonian flows because of the complexity that a non-linear constitutive relationship entails.
It is more practical to approximate non-Newtonian flows using numerical methods.
However, the non-linear constitutive equation--typically of the form $\tau = \mu_{app}(\dot{\gamma}) \dot{\gamma}$, where the apparent viscosity, $\mu_{app}$ is a function of the strain rate--results in certain challenges for numerical methods as well.
%Constitutive form for generalized Newtonian fluids, $\tau = \mu(\dot{\gamma}) \dot{\gamma}$, where the apparent viscosity, $\mu$ is a function of the strain rate
Determining the apparent viscosity and shear rate of a flow will often require an iterative solution, a general Picard-type algorithm is:

\begin{enumerate}
	\item Start with initial guess for the apparent viscosity, $\mu_k = \mu_0$.
	\item \label{step:solve-for-flow} Solve for the flow using the current value of apparent viscosity, $\dot{\gamma}_k = \tau_k / \mu_k$.
	\item Update the apparent viscosity, $\mu_{k+1} = \mu(\dot{\gamma}_k)$.
	\item Return to Step \ref{step:solve-for-flow} until convergence is met.
\end{enumerate}

Numerical solutions work by discretizing the equations that govern the physics of interest.
When the solution for fluid flow problems vary in space and time, a numerical approximation requires breaking the problem up into discrete locations and discrete time steps.
The significance of approximating a solution by discretizing the governing equations is that the iterative solution for the constitutive equation must be solved at each discrete location for each discrete time step, which becomes computationally expensive.

The lattice Boltzmann method (LBM) is a numerical method for fluid flow.
LBM has the unique advantage that computing the strain rate is local to each node, or each discrete location in the domain.
This means that although an iterative solution is still required to determine the local strain rate and apparent viscosity, each iterative solution can be done in parallel, by a separate process, as they are independent of each other.
Because high performance computing is shifting from single, sequential processes to massively parallel processes, the local nature of the stress--strain-rate relationship in LBM gives it a distinct advantage for simulating non-Newtonian flows over other numerical methods for fluid flow. %MG: consider moving literature review on LBM applied to non-Newtonian flows HERE

{\color{red} maybe we can put a brief lit review on LBM applied to non-Newtonian flows here. THEN discuss some drawbacks/instabilities}

LBM does however, have its drawbacks.
LBM can be considered as a type of finite-difference scheme for the continuous Boltzmann equation, and as such, has numerical properties in common with finite-difference schemes. %MG: is this precise enough? accurate enough?
One such consideration associated with this view, is the possibility of numerical inaccuracies and instabilities ~\cite{sterling1993stability,sterling1996stability,bawazeer2013stability,lallemand2000theory}. %MG: are all relevant sources cited?
Much work has been done to investigate and develop ways to enhance the stability of lattice Boltzmann methods.

Sterling and Chen investigated the stability of LBM with the linear Bhatnagar--Gross--Krook (BGK) collision operator on 7-velocity and 9-velocity 2D lattices and a 15-velocity cubic lattice~\cite{sterling1993stability,sterling1996stability}.
The investigation was performed by linearizing all nonlinear terms about global equilibrium values and applying a von Neumann stability analysis on the resulting linearized equations.
The spatial dependence of stability was analyzed by taking the Fourier transform of the linearized equations~\cite{sterling1993stability}.
Sterling and Chen concluded that the linear stability of LBM models depends on the mass distribution parameters, the mean velocity, the relaxation time, and the wave number. %MG: this probably requires some clarification and a more concrete tieback to the highlevel goals of this paper
%(Some misc notes from this study that probably don't fit here but are important to this work: ``... a Courant stability condition is superceded by a more stringent stability condition on the speeds. Another stability boundary common for finite-difference methods requires that the viscous diffusion speed be less than the lattice spacing divided by the time step. This boundary is not observed for LB methods because as the viscosity increases, errors of the scheme increase due to the presence of large nonequilibirum populations, but stability is still maintained''??).

Worthing et. al.~\cite{worthing1997stability} extended the stability analysis performed by Sterling and Chen.
Instead of linearizing the LBM equations by expanding about global equilibrium values, Worthing et. al. \ul{does something else}.
Their analysis discover both physical and nonphysical instabilities, and that setting the BGK relaxation time, $\tau = 1$ provides the optimal accuracy in time. %MG: is this actually relevant though?
\ul{Expand a little more about what they did and its significance here.}
%MG: all this stability talk is nice, but I need to figure out how to tie it back to non-Newtonian flows sooner rather than later. The problem is a lot of these studies focus on instability due to high Reynolds number flows, not really our problem. We're more concerned with instability due to higher relaxation rates and apparent viscosities. And how does MRT address this instability?

Although much work has been done to characterize the stability of LBM models, these analyses have the following limitations to their applicability:

\begin{itemize}
	\item von Neumann stability analysis, and linear stability in general, which is what much of is found in literature, requires modifying the governing equations of the model so that they are linear.
	This linearization cannot be done without introducing some error into the analysis. %MG: linearization is probably not a word
	\item Much of the analysis that exists in literature is done about a zero mean flow~\cite{worthing1997stability}. %MG: definitely cite and/or verify this more
	\item Analyses are most often carried out assuming periodic boundary conditions.
	Periodic boundary conditions are advantageous because a stability analysis is generally conducted by performing a Fourier analysis on the propagation of error and the periodicity of Fourier modes allows the boundary conditions to automatically be satisfied.
	Such an analysis, however, does not generalize well to instances when modes of the system are heavily influenced by the imposed boundary conditions~\cite{worthing1997stability}. %MG: so many questions: like what is an example of an instance when boundary conditions have a major effect and when are they more ``negligible'' or more like periodic BCs?
\end{itemize}

{\color{red} Bawazeer wrote a thesis on the stability and accuracy of the lattice Boltzmann method. Perhaps I'll read it.}

To improve upon the stability of LB models, d'Humi\`{e}res constructed an LBE model with a multiple-relaxation-time (MRT-LB) collision operator, in moment space \cite{d1994generalized}. %MG: make sure you fix the inconsistencies with LB, LBE, etc. I think just use LBE or LBM but think it over. Literature seems to vary.
Lallemand and Luo investigated the dispersion, dissipation, isotropy, Galilean invariance, and stability of the newly constructed LB-MRT method by applying von Neumann stability analysis to it\cite{lallemand2000theory}.
They concluded LBM with the MRT collision operator was more stable than with the BGK collision operator, but with increased computational expense. %MG: I think they have another paper that expands on this study, find and read it.
Note that although this increased computational expense may not be too significant for Newtonian flows (10-20\% \cite{lallemand2000theory}), the issue is magnified for non-Newtonian flows because the iterative solution for the constitutive equation can require that the non-equilibrium distribution function be mapped back and forth from the velocity space to the momentum space multiple times (the strain rate tensor is determined from the non-equilibrium distribution function which must be mapped into momentum space when using the MRT collision operator). %MG: clean this sentence up, and maybe break it down. Too dense, introducing way too many terms the reader hasn't seen before

{\color{red} If you have time to study/implement entropic LBM (very exciting stuff), a brief discussion of its development would go here.}

Our contribution in all of this mess is to investigate the trade-off between computational efficiency and stability/accuracy for the various LBM schemes.
The main advantage of LBM in simulating non-Newtonian flows is the local nature of the stress--strain-rate relationship, resulting in a numerical method that can be run massively in parallel~\cite{something}.
This advantage is much less realized if the collision operator is too computationally expensive, or if instability ensues.
The aim of this paper is to determine approximate numerical values and simulation scenarios in which one LBM method may be more advantageous than another so that scientists and engineers that make use of LBM for simulating non-Newtonian flows can do so in both a computationally efficient and stable manner. %MG: do better than ``simulation scenarios''. what does that even mean?

\section{Lattice Boltzmann Method}

\subsection{Overview}

Some overview of Lattice Boltzmann Method stuff~\cite{grasinger2015simulation}.

\subsection{The Boltzmann Equation}

\subsection{Hermite Polynomial Expansion}

\subsection{Boundary Conditions}

\subsection{Applied Forces}

\subsection{Collision Operator}

\subsubsection{Bhatnagar--Gross--Krook}

\subsubsection{Multiple Relaxation Time}

\subsubsection{Entropic Collisions and Entropy Balance}

{\color{red} This would be sweet if I somehow found the time to study it \cite{gorban2014enhancement}.}

\section{Stability Analysis}

\subsection{von Neumann Stability Analysis}

\subsection{Matrix Method of Stability Analysis}

\subsection{Other Possible Approaches}

\section{Numerical Study}

\subsection{Poiseuille Flow}

\subsection{Lid-driven Flow}

\section{Conclusion}

\section*{Acknowledgements}

\bibliographystyle{abbrv}
\bibliography{master}
	
\end{document}
