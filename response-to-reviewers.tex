\documentclass{article}

\begin{document}
	
	\pagenumbering{gobble}
	
\section*{Response to reviewers}

Thank you very much for your assistance in reviewing this paper. We are sure that with your help, this has become a much better work. We hope that the attached edited copy of ``Numerical Investigation of the Accuracy, Stability, and Efficiency of Lattice Boltzmann Methods in Simulating Non-Newtonian Flow'' will answer any questions previously raised, and will be satisfactory for publication.
The response is followed by a copy of the manuscript with marked up changes for the convenience of any reviewer interested in seeing such a copy.

The following is a list of all comments received, and how they have been addressed in the updated version:

  \subsection*{Reviewer 1}
	\begin{enumerate}
		\item \emph{The authors may test other non-Newtonian fluids more than the present Bingham plastics and power-law fluids.}
		
		Thank you for your comment.
		More complex non-Newtonian behavior would indeed to be interesting to investigate.
		However, as some non-Newtonian constitutive models combine and/or modify Bingham plastic and power-law behavior (e.g. Herschel–Bulkley fluids) these two non-Newtonian fluids were seen as representing the building blocks, in a sense, of more complex non-Newtonian behavior.
		While investigating more non-Newtonian fluids presents interesting future work, we believe that, in the interest of brevity, the current work should remain focused on Bingham plastics and power-law fluids.
		
		\item \emph{How about the non-Newtonian flow in complex geometries,such as the porous media? It is more strict for the stability requirement.}
		
		Thank you for the suggestion.
		Similarly, although the authors agree that such an investigation would be interesting, in the interest of brevity we have chosen not to expand the current investigation beyond Poiseuille flow and lid-driven cavity flow.
		
		\item \emph{An outlook for further models to improve the simulation stability of LBM.}
		
		The discussion of future work at the end of the paper has been expanded to discuss more possible LBM models and stabilization techniques.
	
	\end{enumerate}

	\subsection*{Reviewer 2}
	
	\begin{itemize}
		
	\item \emph{Page 1: The reviewer agrees that artificial dissipation can be an effective technique
		to stabilize a simulation and thus make a certain simulation possible. But the
		reviewer does not agree that it is “effective ... for accurately simulating the flow ...”.
		Especially for non-Newtonian fluids an exact representation of the fluid viscosity is
		essential. By adding numerical dissipation, i.e. viscosity, this is foiled. The Reynolds
		number and thus the flow physics will change and accuracy is decreased. Please
		revise this wording to strengthen that fact.}
	
	Thank you for helping clarify this point. The last sentence of the abstract has been edited in order to more accurately reflect the effect of artificial dissipation on accuracy and stability.
	
	\item \emph{Page 5: Please mention that Eq. (1) is without the effects of external forces}
	
	The sentence prior to the Boltzmann Equation (Eq. (1)) has been edited to clarify this point.
	
	\item \emph{Page 7: ``Due to the discrete nature of velocity in LBM, the integrals simply become
		summations'' Please be more precise, it is an intrinsic part of the discretization
		procedure to replace continuous integrals (moments) by numerical quadrature, i.e.
		Gauss-Hermite quadrature in LBM.}
	
	The aforementioned statement was removed and replaced by explicitly noting that the summations are a Gaussian-type quadrature.
	This was followed by relating the weights of the quadrature to the equilibrium distribution function.
	Additionally, the following connections to numerical quadrature are made directly following Eq. (7) by: (1) mentioning how the weights in Eq. (7) are determined, (2) the relationship between the weights in Eq. (7) and the quadrature weights are given and (3) the reader is referred to some additional reading if they are interested in further detail regarding the numerical quadrature.
	Thank you for this comment.
	The changes mentioned above should help improve the clarity of this paper.
	
	\item \emph{Page 4/15: You mention that computing the strain rate tensor is second order
		accurate, but the reviewer couldn’t find any comment on the accuracy of LBM to
		Navier-Stokes. This could be mentioned in the context with Chapman-Enskog
		expansion (p. 8)}
	
	In the second to last paragraph of page 6, it is mentioned that in the limit of small Mach number and small Knudsen number that the LBE recovers the Navier-Stokes equations.
	Also, the last paragraph of Section 3.1 discusses some of the nuances of the simulation Mach number in the context of simulating non-Newtonian fluids.
	In particular, it mentions that, although reducing the Mach number should increase the accuracy of the LBE with respect to the Navier-Stokes equations, it can introduce error in the constitutive relation.
	
	\item \emph{Page 10: ``A model that has more physical justification but produces unstable and
		nonsensical results is much less useful than a model with some minor artificial
		features yet produces more stable results.'' The reviewer does not agree with this
		statement in general. A stable method, producing ``always'' a result which might have
		significant errors is a dangerous tool...}
	
	The authors agree with this comment and, as such, the above sentenced was removed.
	
	\item \emph{Page 12: Boundary conditions: Please explain here or when describing the setup for
		the numerical calculations if half way or full way bounce back was used}
	
	In Section 2.4 (Boundary Conditions and Applied Forces), it is now explicitly mentioned that the complete bounceback scheme is used for the no-slip boundary condition.
	Additionally, it is made clear that, in general, boundary conditions and the incorporation of external forces in LBM is not unique.
	
	\item \emph{Page 15: Setup of Bingham Poiseuille flow: What is the Mach number in the lattice
		for the simulations?}
	
	A column has been added to each table to show the Mach number of each simulation.
	
	\item \emph{Page 15: An exemplary figure (e.g. for $\tau_y = 4.0*10^-5$) comparing the viscosity
		according to the exact Bingham relationship and the approximations according to
		Papanastasiou with $m=10^5$ and $m=10^8$, respectively, would allow a comparison at
		first glance. Also the reader could see differences directly.}
	
	This figure was added.
	Thank you for the suggestion.
	
	\item \emph{Page 22: Why don’t you compare the MRT calculations to the BGK,
		$m=10^8$, cases, since $m=10^8$ was used for MRT as well?}
	
	This is a good point.
	The discussion has been expanded to compare the MRT calculations to the BGK, $m = 10^8$ cases.
	
	\item \emph{Page 23, 24: You draw some conclusions between the ``ghost'' modes ($\epsilon$ and $q$) and
		nonphysical oscillations. For example you state (page 23): ``that the nonphysical
		oscillations the BGK with $m = 10^8$ displayed was in part due to the $\epsilon$ moment''.
		To the reviewer there is no direct proof visible for these conclusions. It might also be
		the other way around; oscillations in the distributions cause the oscillations in the
		moments. Please explain this in more detail.}
		
	The authors agree with the reviewer.
	The evidence is inconclusive and as such one can only make loose inferences on the actual cause of oscillations.
	The language in the discussion in pages 23-29 has been altered to reflect the fact that the results do not offer direct proof.
	In addition, parts of the discussion has been expanded to better motivate and explain why the authors believe the nonequilibrium $\epsilon$ and $q_x$ moments may be the primary source of oscillations.
	
	\item \emph{Page 23, 24: Did you investigate if there are oscillations in the moment e, related to
		the bulk viscosity? It is well known that a high bulk viscosity is stabilizing and thus
		producing some of the superior MRT stability properties.}
	
	The nonequilibrium $e$ moment was indeed calculated and investigated.
	It has been added to the discussion of the nonequilibrium moments.
	
	\item \emph{Page 25: ``...ghost mode associated with the q x moment dominates and is the
		primary source of oscillations.'' Again, this is not obvious for the reviewer, please
		prove this conclusion.}
	
	Again, the authors agree that direct proof cannot be found in the results.
	The language has been changed to reflect this.
	In addition, parts of the discussion has been expanded to better motivate and explain why the authors believe the nonequilibrium $\epsilon$ and $q_x$ moments may be the primary source of oscillations.
	
	\item \emph{Page 25: ``Overall, the results presented suggest that if an LBM collision scheme is to
		be developed ... , it should focus on a means of dampening the nonequilibrium
		energy flux moments, namely q x and q y''. Did you consider the two relaxation times
		model (TRT) of Ginzburg et al.?}
	
	The two relaxation time model was not investigated. However, the discussion of possible future work has been expanded to included the TRT.
	
	\item \emph{Page 25/27: Is the lattice number (128x32) correct? Since you reduced the number
		of nodes along the channel height from 64 to 32? What was the Mach number for
		these simulations? The error of all MRT results is drastically increased in Table 2
		compared to Table 1 what fits to the reduced lattice number in along the channel
		height. Can you explain this, also in regard to the behavior of the BGK error?}
	
	The lattice number was incorrect (it should be 32x128).
	Thank you for pointing out this mistake.
	The Mach number of these simulations were added to the table.
	Additionally, a paragraph was added to the end of Section 3.1 that aims to explain why the MRT results are less accurate despite being simulated with a finer lattice resolution.
	
	\item \emph{Page 27: Lid driven cavity: Please explain the setup of the simulation in more detail:
		e.g. boundary conditions, lattice Mach number, ...}
	
	The setup of the simulations is given in more detail.
	
	\item \emph{Page 30: Some exemplary figures of the flow pattern in the cavity, e.g. streamlines or
		contour plots, would give more insight, especially about smoothness or oscillations in
		the solution}
	
	Streamlines comparing the flow patterns of the various collision schemes for some of the more interesting non-Newtonian fluids and Reynolds numbers are now provided and accompanied by discussions of the results.
	Thank you for the suggestion.
	This should give a clearer picture of the similarities and differences between the solutions of the various collision schemes.
	
	\item The typos that you mentioned in your review have also been corrected.
	
	\end{itemize}

Thank you again for your help. Please let us know of any further improvements that can be made to this paper.

\vspace{6in}
\noindent Sincerely, \\
\indent John C. Brigham \vspace{0.1in}\\
\indent Associate Professor \\
\indent School of Engineering and Computing Sciences
	
\end{document}
